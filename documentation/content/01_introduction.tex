\chapter{Introduction}

The last several years brought revolution in positioning and geospatial technologies, enabling enterprises to boost their core businesses with the potentials these technologies are bringing. One of the adoptions are Location-Based Services (LBS), which take advantage of mobile devices, wireless, GSM and GPS technologies altogether. Due to the amount of data these positioning technologies generate, companies started looking at efficient ways of storing and processing the data - such as in S3 compatible storage, NoSQL databases, Apache Spark processing engines, using microservices architecture.

The idea behind this project is to provide algorithms and architectural software solutions which will be able to process big amounts of data, visualize it and give macroscopic view on the collected data across many users. Particularly, we look at where the users stop and what are the significant places considering their movements in a macroscopic city view. 

Location Based Services have an advantage that users do not have to enter their position manually, it is automatically collected (using all positioning technologies available in the mobile device), and used to generate personalized information. This service is of significant value for stores and shopping facilities, airports, tourism, car-sharing, payment systems, and more. Positioning technologies, being a core for LBS, are a trade off between accuracy of the positioning and its applicability indoor/outdoor. Bluetooth and Wifi technologies provide highest positioning accuracy, however they can only be used indoors. On the other hand, GPS that also providing relatively good accuracy can only be used outdoors. Cellular, GSM networks can be used indoors and outdoors, however they provide moderate to low accuracy (macroscopic picture) \cite{LocPos1}. The are two types of LBS, proactive services which requests whenever user enters certain geographic area (approaches a shop or service he might be interested in or when he leaves the shared-car he has been renting), or reactive, in which user explicitly requests to use service (to find nearest point of interest). Thus, for most application, accuracy of GSM or WiFi/Bluetooth network is more then enough, and GSM is not playing the major role \cite{LocPos2}. Application using LBS, at the given instance of time will try to obtain the location using the positioning technique which is currently available and having highest accuracy at that time. This implies, that the data collected cannot be considered as the one representing the exact location of the user at the specific moment in time, but only gives information, that user has entered/moved to a new geographical area in case of proactive LBS or that user is somewhere in the specified area in case of reactive LBS. 